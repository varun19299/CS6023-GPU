\section {Implementation and Technical Notes}

The code was run on a \textit{GTX 1080 Ti} and not the cluster of \textit{K'40s}. This could be a reason for variance in the optimal configuration found for the GPUs. However, the code does not make any hardware assumptions and can therefore be run on any suitable cluster. \\

CUDA version 9.0 was used to compile the code, with \lstinline{nvcc} as the device compiler and \lstinline{gcc} as the host compiler.

\section {Question 1 and 2}

Matrix Multiplication via thread-wise access to global memory was used.\\

Following parameters were queried:
\begin{itemize}
\item Scope of support of L1 Cache (Global or local): \quad \textbf{Yes}
\item Size of L2 Cache: \quad \textbf{Yes}
\item Maximum permissible threads per block: \quad \textbf{1024}
\item Registers allocated per block: \quad \textbf{65536}
\item Registers available in a streaming multiprocessor: \quad \textbf{65536}
\item Warp Size (bytes) : \quad \textbf{32}
\item Total amount of memory available in the GPU (in bytes): \quad \textbf{11719409664} (12GB)
\end{itemize}

\subsection {Code Blocks (pertinent only)}
\begin{lstlisting}
_global__ void MatrixMulKernel_col_maj(double* M, double* N, double* P, int Width) { 
    // Calculate the row index of the P element and M
    int Row = blockIdx.y*blockDim.y+threadIdx.y;
    // Calculate the column index of P and N
    int Col = blockIdx.x*blockDim.x+threadIdx.x; 
    
    if ((Row < Width) && (Col < Width)) {
            float Pvalue = 0;
        for (int k = 0; k < Width; ++k) {
            Pvalue += M[Row*Width+k]*N[k*Width+Col];
        }
            P[Row*Width+Col] = Pvalue;
        }
    }

__global__ void MatrixMulKernel_row_maj(double* M, double* N, double* P, int Width) { 
    // Calculate the row index of the P element and M
    int Row = blockIdx.y*blockDim.y+threadIdx.x;
    // Calculate the column index of P and N
    int Col = blockIdx.x*blockDim.x+threadIdx.y; 
    
    if ((Row < Width) && (Col < Width)) {
            float Pvalue = 0;
        for (int k = 0; k < Width; ++k) {
            Pvalue += M[Row*Width+k]*N[k*Width+Col];
        }
            P[Row*Width+Col] = Pvalue;
        }
    }
\end{lstlisting}

We run two kernels, one with the fastest varying index as
\bigskip
\subsection {Roof-line Plot}
\noindent Plot of the roof-line model for the queried GPU.\\

The Roof-line model helps understand the bottlenecks in achieving greater performance. Till a certain operational intensity, it is limited by the bandwidth of global context transfer (host to device data). Further, there is the peak \textbf[TFLOPS] limit, which is a hardware-architecture limit.\\

For the case of the \textit{GeForce 1080 Ti}, the peak FLOPS may be computed as:
$$ Cores * FLOPS *Clock Frequency$$
This yields 11.3 T FLOPS for the GPU, which matches closely with the manufacturer's benchmarks (11.5). Peak data bandwidth is obtained from the device query.

The resultant plot : \textbf{Figure 1}

\begin{figure}[ht]
\centering
%%\includegraphics[angle=0,width=0.8\textwidth]{Figure_1.png}
\caption{Roof-line model for \textbf{GTX 1080 Ti}}
\end{figure}

  \subsection{Other Details}
Device Number: 0
  
  
  \begin{table}[ht]
\footnotesize
\centering
\begin{tabular}{r||rrr}
Parameter & Specifics  \\ 
 \hline \hline
Device name & GeForce GTX 1080 Ti \\
  Memory Clock Rate (KHz) & 5505000 \\
  Memory Bus Width (bits) & 352 \\
  Is L1 Cache supported globally & Yes \\
  Is L1 Cache supported locally & Yes \\
  L2 Cache Size (bytes) & 2883584 \\
  Max no of threads per block & 1024 \\
  No of registers available in a block & 65536 \\
  No of registers available in a streaming multiprocessor & 65536 \\
  Warp Size (bytes)&  32 \\
  Grid Size (bytes) & 140731727872496 \\
  Total memory (bytes) & 11719409664 \\
  Peak Memory Bandwidth (GB/s) & 484.440000 \\
\end{tabular}
\caption{Device specifications, \textit{device:0}}
\end{table}

\newpage

\section {Question 3 and 4}
\noindent We find the optimal number of threads per block to run the given vector addition empirically. \\

\textbf{Question 3} involves writing kernel functions to implement vector addition over multiple threads, one operation each thread. We output the result of adding two $2^{15} $sized vectors, which are randomly generated. \\

For \textbf{Question 4}, we empirically determine the optimal number of threads per block. From \textbf{Question 1} we know that the maximum permissible threads per block is 1024, hence we vary the threads per block from 128 to 1024 in powers of 2, and calculate the run-time in each case. \\

Note that, the run time has been calculated without considering the host allocation time. We believe that this is a valid choice, since the host allocation time maybe isolated for each process and therefore can be effectively excluded from the run time calculations.\\

\subsection{Code Changes}
\begin{lstlisting}
	__global__ void VecAdd(float* A, float* B, float* C, int N){
      // Host code
      int i = blockDim.x * blockIdx.x + threadIdx.x;
      if (i < N)
          C[i] = A[i] + B[i];
    }
    // .... //
    
    printf("Array A (first 10 values) \n ");
    for(loop = 0; loop < N; loop++){
    h_A[loop] = rand() % 100 + 1;
      if (loop<10){
          printf("%f ", h_A[loop]);
      }
    }

    printf("\nArray B (first 10 values) \n ");
    for(loop = 0; loop < N; loop++){
        h_B[loop] = rand() % 100 + 1;
        if (loop<10){
            printf("%f ", h_B[loop]);
       }
     }
// .... //
cudaEventRecord(start, 0);
VecAdd<<<blocksPerGrid, threadsPerBlock>>>(d_A,d_B, d_C, N);
cudaEventRecord(stop, 0);
cudaEventSynchronize(stop);
cudaEventElapsedTime(&time_spent, start, stop);
time_spent=time_spent/(avg_loop-1)*10;
\end{lstlisting} \\

\begin{table}[ht]
\footnotesize
\centering
\begin{tabular}{r||rrr}
 SI (vector addition of $2^{15}$ size) & Threads per Block (powers of 2) & Average GPU run-time (for 10 runs)  \\ 
 \hline \hline
1 & 32 &  0.312521 \\ 
2& 64  & 0.304183 \\
3& 128 & 0.304020 \\
4& 256 & 0.303870 \\
5& 512 & 0.300294 \\
6 & 1024 &0.286491 \\
\end{tabular}
\caption{Average time for 10 passes versus threads per block}
\end{table}

The consequent plot: \textbf{Figure 2}

\begin{figure}[ht]
\centering
%%\includegraphics[angle=0,width=0.7\textwidth]{Figure_2.png}
\caption{Average Runtime versus Threads per Block for \textbf{GTX 1080 Ti}}
\end{figure}

Hence, \textbf{optimal no of threads per block} is \textbf{1024}.\\

\subsection{Reasoning}

The optimal threads per block may not be upper bounded for this low intensity (operational), and hence, using the maximum possible number of threads per block is optimal.
\section {Question 5}

Here, we empirically investigate the dependence of run-time on the number of operations per thread.

\subsection{Code Changes} 
\begin{lstlisting}
	__global__ void VecAdd(float* A, float* B, float* C, int N){
      // Host code
      int j;
    int i = blockDim.x * blockIdx.x + threadIdx.x;
    if (i < N_op){
        for (j=0;j<op_loop;j++)
            C[i*op_loop+j] = A[i*op_loop+j] + B[i*op_loop+j];
    }
    }
   // Array of op's to try//
   for (op_loop_ii=0;op_loop_ii<10;op_loop_ii++){
        op_loop_array[op_loop_ii]=pow(2,op_loop_ii);
    }
// Run kernel over these ops and average each run //
cudaEventRecord(start, 0);
VecAdd<<<blocksPerGrid, threadsPerBlock>>>(d_A,d_B, d_C, N);
cudaEventRecord(stop, 0);
cudaEventSynchronize(stop);
cudaEventElapsedTime(&time_spent, start, stop);
time_spent=time_spent/(avg_loop-1)*10;
\end{lstlisting} \\

\subsection{Run-Times}

\textbf{Table 3} contains the relevant logs.

\begin{table}[ht]
\footnotesize
\centering
\begin{tabular}{r||rrr}
 SI (vector addition of $2^{15}$ size) & Ops per loop (powers of 2) & Average GPU run-time (for 10 runs)  \\ 
 \hline \hline
1 & 1 &  0.298492 \\ 
2& 2  & 0.349331 \\
3& 4 & 0.296480 \\
4& 8 & 0.340138 \\
5& 16 & 0.374680 \\
6 & 32 &0.454981 \\
7 & 64 &0.570481 \\
8 & 128 &0.864831 \\
\end{tabular}
\caption{Average time for 10 passes versus ops per thread}
\end{table}

The consequent plot: \textbf{Figure 3} \\

\begin{figure}[ht]
\centering
%%\includegraphics[angle=0,width=0.7\textwidth]{Figure_3.png}
\caption{Average Runtime versus Ops per Thread for \textbf{GTX 1080 Ti}}
\end{figure}

Hence, \textbf{optimal no of ops per thread} is \textbf{4}.\\

\subsection{Reasoning}

The optimal operations per block is dependent on the level of SIMD parallelism a single thread may be able to achieve. With an optimal of \textbf{4}, it is possible that the device has over \textbf{8 ALU's} per thread-context. \\

\section {Question 6 and 7}
Again, empirically, we observe the changes in run-time per vector size of the random vectors being operated upon.
 
 \subsection{Run-Times}
 
 No major code changes here, besides varying the vector sizes given these optimal \textit{Threads per Block} and \textit{operations per thread}. \\

\begin{table}[ht]
\footnotesize
\centering
\begin{tabular}{r||rrr}
 SI (vector addition of $2^{15}$ size) & Vector Size (powers of 2) & Average GPU run-time ( averaged over 1000 runs)  \\ 
 \hline \hline
1 & $2^{15}$ &  0.032892 \\ 
2& $2^{16}$   & 0.032452 \\
3&  $2^{17}$ & 0.044452 \\
4&  $2^{18}$ & 0.101014 \\
5&  $2^{19}$ & 0.198954 \\
6 &  $2^{20}$ &0.341814 \\
\end{tabular}
\caption{Average time for single pass versus Vector Size}
\end{table}
 
 Consequent plot: \textbf{Figure 4}
 
 \begin{figure}[ht]
\centering
%%\includegraphics[angle=0,width=0.7\textwidth]{Figure_4.png}
\caption{Average Runtime versus Vector Size for \textbf{GTX 1080 Ti}}
\end{figure}


\subsection{Reasoning}

The variation of runtimes with size is not linear $O(n)$ as would have been expected of a CPU, but is more or less constant upto $2^{18}$ as a vector size. Post this, it is possible that the size no longer allows the same degree of parallelism with respect to caching, Grid-to-thread transfer, etc.
 
